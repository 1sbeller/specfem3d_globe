\chapter*{Copyright}\label{cha:Copyright}
\addcontentsline{toc}{chapter}{Copyright}

Main `historical' authors: Dimitri Komatitsch and Jeroen Tromp (there are now many more!).

Princeton University, USA, and CNRS / University of Marseille, France.

$\copyright$ Princeton University, USA and CNRS / University of Marseille, France, April 2014

This program is free software; you can redistribute it and/or modify
it under the terms of the GNU General Public License as published
by the Free Software Foundation (see Appendix \ref{cha:License}).\\

\textbf{\underline{Evolution of the code:}}\\

 v. 7.0, many developers, January 2015:
     simultaneous MPI runs, ADIOS file I/O support, ASDF seismograms, new seismogram names, tomography tools,
     CUDA and OpenCL GPU support, CEM model support, updates AK135 model, binary topography files,
     fixes geocentric/geographic conversions, updates ellipticity and gravity factors, git versioning system.\\

 v. 6.0, Daniel Peter (ETH Z\"urich, Switzerland), Dimitri Komatitsch and Zhinan Xie (CNRS / University of Marseille, France),
     Elliott Sales de Andrade (University of Toronto, Canada), and many others, in particular from Princeton University, USA,
     April 2014:
     more flexible MPI implementation, GPU support, exact undoing of attenuation, LDDRK4-6 higher-order time scheme, etc...\\

 v. 5.1, Dimitri Komatitsch, University of Toulouse, France and Ebru Bozdag, Princeton University, USA, February 2011:
     non blocking MPI for much better scaling on large clusters;
     new convention for the name of seismograms, to conform to the IRIS standard;
     new directory structure.\\

 v. 5.0, many developers, February 2010:
     new Moho mesh stretching honoring crust2.0 Moho depths,
     new attenuation assignment, new SAC headers, new general crustal models,
     faster performance due to Deville routines and enhanced loop unrolling,
     slight changes in code structure (see also trivia at program start).\\

 v. 4.0 David Mich\'ea and Dimitri Komatitsch, University of Pau, France, February 2008:
      first port to GPUs using CUDA, new doubling brick in the mesh, new perfectly load-balanced mesh,
      more flexible routines for mesh design, new inflated central cube
      with optimized shape, far fewer mesh files saved by the mesher,
      global arrays sorted to speed up the simulation, seismograms can be
      written by the master, one more doubling level at the bottom
      of the outer core if needed (off by default).\\

 v. 3.6 Many people, many affiliations, September 2006:
      adjoint and kernel calculations, fixed IASP91 model,
      added AK135F\_NO\_MUD and 1066a, fixed topography/bathymetry routine,
      new attenuation routines, faster and better I/Os on very large
      systems, many small improvements and bug fixes, new `configure'
      script, new Pyre version, new user's manual etc..\\

 v. 3.5 Dimitri Komatitsch, Brian Savage and Jeroen Tromp, Caltech, July 2004:
      any size of chunk, 3D attenuation, case of two chunks,
      more precise topography/bathymetry model, new Par\_file structure.\\

 v. 3.4 Dimitri Komatitsch and Jeroen Tromp, Caltech, August 2003:
      merged global and regional codes, no iterations in fluid, better movies.\\

 v. 3.3 Dimitri Komatitsch, Caltech, September 2002:
      flexible mesh doubling in outer core, inlined code, OpenDX support.\\

 v. 3.2 Jeroen Tromp, Caltech, July 2002:
      multiple sources and flexible PREM reading.\\

 v. 3.1 Dimitri Komatitsch, Caltech, June 2002:
      vectorized loops in solver and merged central cube.\\

 v. 3.0 Dimitri Komatitsch and Jeroen Tromp, Caltech, May 2002:
   ported to SGI and Compaq, double precision solver, more general anisotropy.\\

 v. 2.3 Dimitri Komatitsch and Jeroen Tromp, Caltech, August 2001:
                       gravity, rotation, oceans and 3-D models.\\

 v. 2.2 Dimitri Komatitsch and Jeroen Tromp, Caltech, USA, March 2001:
                       final MPI package.\\

 v. 2.0 Dimitri Komatitsch, Harvard, USA, January 2000: MPI code for the globe.\\

 v. 1.0 Dimitri Komatitsch, UNAM, Mexico, June 1999: first MPI code for a chunk.\\

 Jeroen Tromp and Dimitri Komatitsch, Harvard, USA, July 1998: first chunk solver using OpenMP on a Sun machine.\\

 Dimitri Komatitsch, IPG Paris, France, December 1996: first 3-D solver for the CM-5 Connection Machine,
    parallelized on 128 processors using Connection Machine Fortran.\\

