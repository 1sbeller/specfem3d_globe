\chapter*{Simulation features supported in SPECFEM3D\_GLOBE}
\addcontentsline{toc}{chapter}{Simulation features supported in SPECFEM3D\_GLOBE}

The following lists all available features for a SPECFEM3D\_GLOBE simulation,
where {\it CPU}, {\it CUDA} and {\it OpenCL} denote the code versions for CPU-only simulations,
CUDA and OpenCL hardware support, respectively.
%
\begin{table}[htp]
\label{table:features}
\begin{center}
\begin{tabular}{ l l c c c}
\hline
%{\bf Feature}    &   & \multicolumn{3}{c}{{\bf Code version}} \\
%\cmidrule(lr){3-5}
%           &     & {\it CPU} & {\it CUDA}  & {\it OpenCL} \\
%% to have proper title w/ pandoc
{\bf Feature}   &   & {\it CPU} & {\it CUDA}  & {\it OpenCL} \\
\hline
& & & & \\
%%
{\bf Physics}       & Ocean load      & X   & X   & X \\
                & Ellipticity               & X   & X       & X \\
                & Topography        & X       & X       & X \\
                & Gravity               & X       & X       & X \\
                & Rotation              & X       & X       & X \\
                & Attenuation       & X       & X       & X \\
\hline
& & & & \\
%%
{\bf Simulation Setup}  & Global (6-chunks)     & X & X & X \\
                  & Regional (1,2-chunk)    & X & X & X \\
                  & Restart/Checkpointing & X & X & X \\
                  & Simultaneous runs     & X & X & X \\
                  & ADIOS file I/O        & X & X & X \\
\hline
& & & & \\
%%
{\bf Sensitivity kernels} & Partial physical dispersion     & X     & X     & X \\
                  & Undoing of attenuation          & X     & X     & X \\
                  & Anisotropic kernels               & X     & X     & X \\
                  & Transversely isotropic kernels    & X     & X     & X \\
                  & Isotropic kernels             & X     & X     & X \\
                  & Approximate Hessian           & X     & X     & X \\
\hline
& & & & \\
%%
{\bf Time schemes}      & Newmark     & X     & X     & X \\
                    & LDDRK     & X     & -     & - \\
\hline
& & & & \\
%%
{\bf Visualization}     & Surface movie   & X     & X     & X \\
                  & Volumetric movie  & X     & X     & X \\
\hline
& & & & \\
%%
{\bf Seismogram formats}  & Ascii           & X   & X   & X \\
                    & SAC       & X & X & X \\
                    & ASDF      & X & X & X \\
                    & Binary      & X & X & X \\
%
\hline
& & & & \\ % to avoid clashes with pandoc
\end{tabular}
\end{center}
\end{table}


