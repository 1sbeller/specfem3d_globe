\chapter{Troubleshooting}\label{cha:Troubleshooting}

\section*{FAQ}

\begin{description}
\item [configuration fails:]
   Examine the log file 'config.log'. It contains detailed informations.
   In many cases, the path's to these specific compiler commands F90,
   CC and MPIF90 won't be correct if `./configure` fails.

   Please make sure that you have a working installation of a Fortran compiler,
   a C compiler and an MPI implementation. You should be able to compile this
   little program code:
\begin{lyxcode}
{\footnotesize      program main }{\footnotesize \par}
{\footnotesize        include 'mpif.h' }{\footnotesize \par}
{\footnotesize        integer, parameter :: CUSTOM\_MPI\_TYPE = MPI\_REAL }{\footnotesize \par}
{\footnotesize        integer ier }{\footnotesize \par}
{\footnotesize        call MPI\_INIT(ier) }{\footnotesize \par}
{\footnotesize        call MPI\_BARRIER(MPI\_COMM\_WORLD,ier) }{\footnotesize \par}
{\footnotesize        call MPI\_FINALIZE(ier) }{\footnotesize \par}
{\footnotesize      end}{\footnotesize \par}
\end{lyxcode}


\item [compilation fails:] In case a compilation error like the following occurs, stating
\begin{lyxcode}
{\footnotesize    ...  }{\footnotesize \par}
{\footnotesize    obj/meshfem3D.o: In function `MAIN\_\_':  }{\footnotesize \par}
{\footnotesize    meshfem3D.f90:(.text+0x14): undefined reference to `\_gfortran\_set\_std'  }{\footnotesize \par}
{\footnotesize    ...  }{\footnotesize \par}
\end{lyxcode}
  make sure you're pointing to the right 'mpif90' wrapper command.

  Normally, this message will appear when you are mixing two different Fortran
  compilers. That is, using e.g. gfortran to compile non-MPI files
  and mpif90, wrapper provided for e.g. ifort, to compile MPI-files.

  fix: e.g. specify > ./configure FC=gfortran MPIF90=/usr/local/openmpi-gfortran/bin/mpif90


\item [changing PPM model routines fails:]
  In case you want to modify the PPM-routines in file \texttt{model\_ppm.f90}, please consider the following points:

  \begin{enumerate}
  \item Please check in file \texttt{get\_model\_parameter.f90} that the entry for PPM models looks like:
  \begin{lyxcode}
{\footnotesize  ... }{\footnotesize \par}
{\footnotesize  else if(MODEL\_ROOT == 'PPM') then  }{\footnotesize \par}
{\footnotesize   ! overimposed based on isotropic-prem  }{\footnotesize \par}
{\footnotesize   CASE\_3D = .true.  }{\footnotesize \par}
{\footnotesize   CRUSTAL = .true.  }{\footnotesize \par}
{\footnotesize   ISOTROPIC\_3D\_MANTLE = .true.  }{\footnotesize \par}
{\footnotesize   ONE\_CRUST = .true.  }{\footnotesize \par}
{\footnotesize   THREE\_D\_MODEL = THREE\_D\_MODEL\_PPM  }{\footnotesize \par}
{\footnotesize   TRANSVERSE\_ISOTROPY = .true. ! to use transverse-isotropic prem  }{\footnotesize \par}
{\footnotesize  ...  }{\footnotesize \par}
  \end{lyxcode}
  You can set \texttt{TRANSVERSE\_ISOTROPY} to \texttt{.false.} in case you want to use the isotropic PREM
  as 1D background model.

  \item Transverse isotropy would mean different values for horizontal and vertically polarized wave speeds,
  i.e. different for vph and   vpv, vsh and vsv, and it includes an additional parameter eta.
  By default, we take these wave speeds from PREM and add your model perturbations to them.
  For the moment, your model perturbations are added as isotropic perturbations, using the same dvp for vph and vpv,
  and dvs for vsh   and vsv, see in \texttt{meshfem3D\_models.f90}:
  \begin{lyxcode}
{\footnotesize  ... }{\footnotesize \par}
{\footnotesize     case(THREE\_D\_MODEL\_PPM ) }{\footnotesize \par}
{\footnotesize       ! point profile model }{\footnotesize \par}
{\footnotesize       call model\_PPM(r\_used,theta,phi,dvs,dvp,drho) }{\footnotesize \par}
{\footnotesize       vpv=vpv*(1.0d0+dvp) }{\footnotesize \par}
{\footnotesize       vph=vph*(1.0d0+dvp) }{\footnotesize \par}
{\footnotesize       vsv=vsv*(1.0d0+dvs) }{\footnotesize \par}
{\footnotesize       vsh=vsh*(1.0d0+dvs) }{\footnotesize \par}
{\footnotesize       rho=rho*(1.0d0+drho) }{\footnotesize \par}
{\footnotesize  ... }{\footnotesize \par}
  \end{lyxcode}
 You could modify this to add different perturbations for vph and vpv, resp. vsh and vsv.
 This would basically mean that you add transverse isotropic perturbations.
 You can see how this is done with e.g. the model \texttt{s362ani},
 following the flag \texttt{THREE\_D\_MODEL\_S362ANI} on how to modify accordingly the file \texttt{meshfem3D\_models.f90}.

  In case you want to add more specific model routines, follow the code sections starting with:
  \begin{lyxcode}
{\footnotesize  !--- }{\footnotesize \par}
{\footnotesize  ! }{\footnotesize \par}
{\footnotesize  ! ADD YOUR MODEL HERE }{\footnotesize \par}
{\footnotesize  ! }{\footnotesize \par}
{\footnotesize  !--- }{\footnotesize \par}
  \end{lyxcode}
  to see code sections sensitive to model updates.

  \end{enumerate}

\end{description}



